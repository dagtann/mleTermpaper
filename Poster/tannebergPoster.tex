\documentclass[landscape,paperheight=24in,fontscale=.45,paperwidth=36in]{baposter}

\usepackage{blindtext}

\usepackage{amsmath}
\usepackage{amssymb}
\usepackage{amsfonts}
\usepackage{bbm}                     %% indicator function symbol

\usepackage{graphicx}
\usepackage{caption}                           %% custom captions
\usepackage{soul}                          %% colorful underlines

%% --- Define WZB color scheme ----------------------------------
\definecolor{wzbGreen}{cmyk}{0.366,0,0.967,0.4}
\definecolor{wzbBlue}{cmyk}{0.983,0.293,0,0.29}
\definecolor{wzbMagenta}{cmyk}{0,0.69,0.272,0.38}

%% --- Experimental colors --------------------------------------
\definecolor{pineGreen}{cmyk}{0.9833,0,0,0.5294}
\definecolor{claret}{cmyk}{0,0.8281,0.5625,0.4980}
\definecolor{threequartergray}{gray}{0.75}
\definecolor{halfgray}{gray}{0.5}
\definecolor{quartergray}{gray}{0.25}

%% --- define underline color -----------------------------------
\setulcolor{wzbBlue}

\begin{document}
\begin{poster}{
  % --- General options ------------------------------------
  eyecatcher=true,  %% No funny picture to left
  background=none, %% white background
  % --- Box header settings --------------------------------
  headershape=rounded, %% shape box header
  headerborder=open, %% add border to header
  headerColorOne=wzbBlue, %% bg color header
  %headerColorTwo=wzbBlue,
  headershade=plain, %% no color gradient header
  % --- Box settings ---------------------------------------
  boxshade=none, %% no color gradient box content
  borderColor=wzbBlue, %% border color box content
  textborder=roundedright %% bottom shape box border
}
{ % Worlmap displaying dictatorships in 2007
 \includegraphics[scale=.6]{/home/dag/Dropbox/Buero/Dissertation/2015/duke/classes/mle/termPaper/out/worldmapEyecatcher.pdf}
}
{\sf A dictator's toolkit. How cooptation affects repression in autocracies}
{
\textsf{Replicated by} Dag Tanneberg, Berlin Social Science Center,
  Dept. `Democracy and Democratization', dag.tanneberg@wzb.eu%
}
{
\includegraphics[scale=.5]{./quareise.png}
}
\headerbox{\sf \color{white} Summary}{name=summary,column=0,row=0}{
\begin{minipage}{\linewidth}
  \raggedright Frantz and Kendall-Taylor explore the interaction 
  of two instruments of dictatorial rule: co-optation and 
  political repression. They find that co-optation in the form of
  parties and legislatures leads dictators to reduce 
  restrictions on empowerment rights. Simultaneously, it increases
  physical integrity violations. This replication finds 
  evidence of model misspecification and weak predictive 
  accuracy. An extension emphasizes the interaction of co-optation 
  and repression.
\end{minipage}
}

\headerbox{\sf \color{white} Method and material}{name=methods,column=0,below=summary}{
  \begin{minipage}{\linewidth}
  \raggedright
  Ordered logistic regression: $\xi = \alpha + \boldsymbol{x}\boldsymbol{\beta}+\mathcal{\varepsilon}$
  \begin{align*}
    & y = \begin{cases}
      \text{level}~ 1 ~ \text{if} -\infty < \xi \le \alpha_1 \\
      \vdots \\
      \text{level}~ m-1 ~ \text{if} ~ \alpha_{m-2} < \xi \le \alpha_{m-1} \\
      \text{level}~ m ~ \text{if} ~ \alpha_{m-1} < \xi \le \infty \\
    \end{cases} \\
    & Pr(y \le j | \boldsymbol{x}) = Pr(\xi \le \alpha_j|\boldsymbol{x})
  \end{align*}
  Design: 138 dictatorships between 1972 and 2007 are analyzed 
  in a pooled cross-section design with five imputed datasets. The 
  study explores the effect of co-optation at $t$ on repression at
  $t+1$ to $t+5$.
  \end{minipage}
}

\headerbox{\sf \color{white} First impression}{name=first,column=0,below=methods}{
  \begin{minipage}{\linewidth}
  \centering
  \includegraphics[width=3.6in]{/home/dag/Dropbox/Buero/Dissertation/2015/duke/classes/mle/termPaper/out/coefPlotOriginal.pdf} \\
  \raggedright
  The original results can be reproduced up to two decimal places. 
  Co-optation via legislatures and political parties {\color{wzbBlue}
  reduces} restrictions on empowerment rights. Simultaneously, it 
  {\color{wzbMagenta} encourages} physical integrity violations.
  \end{minipage}
}
\headerbox{\sf \color{white} Replication results}{name=results,column=1,span=2,row=0,textborder=rounded,bottomaligned=first}{
\centering{\section*{\sf \underline{Parallel regressions assumption}}}
\begin{minipage}{.49\linewidth}
  \centering
  \includegraphics[width=3in]{/home/dag/Dropbox/Buero/Dissertation/2015/duke/classes/mle/termPaper/out/parallelRegrDevianceBonfP.pdf} \\
  \raggedright
  Using a $\chi^2$-test the coefficients from each imputation
  round can be compared to their multinomial alternatives. Only 
  four models reject the multinomial alternative hypothesis of 
  non-constant coefficients. \newline
\end{minipage}
\hfill
\begin{minipage}{.49\linewidth}
  \centering
  \includegraphics[width=3in]{/home/dag/Dropbox/Buero/Dissertation/2015/duke/classes/mle/termPaper/out/parallelRegressionsPosterPlot_ManualLabels.pdf} \\
  \raggedright
  An alternative test draws on $j-1$ logistic regressions with 
  response $\mathbbm{1}_y(y_{i} \ge j)$. Coefficients should differ 
  little as $j$ increases. Perfect separation occurred 
  for some response levels. The test raises strong 
  concerns about co-optation. \newline
\end{minipage}
\vfill
\vfill
%% --- Separation plots ----------------------------------------
\begin{minipage}{.49\linewidth}
  \centering
  \section*{\sf \underline{Predictive accuracy}}
  \includegraphics[width=3in]{/home/dag/Dropbox/Buero/Dissertation/2015/duke/classes/mle/termPaper/out/separation_revise.pdf} \\
  \raggedright
  As can be seen from the separation plots empowerment rights 
  restrictions at $t+1$ are somewhat reliably predicted.
  Already at $t+2$ predictive accuracy declines visibly. Moreover, 
  co-optation offers no leverage on physical integrity 
  violations.
\end{minipage}
\hfill
\begin{minipage}{.49\linewidth}
  \bigskip
  \centering
  \section*{\sf \underline{Parsimony}}
  \includegraphics[width=3in]{/home/dag/Dropbox/Buero/Dissertation/2015/duke/classes/mle/termPaper/out/aicDifferences.pdf} \\
  \raggedright
  All models include lagged responses to account for serial 
  autocorrelation. However, in many cases the lagged response 
  offers the most parsimonious fit to the data. Again co-optation
  offers no leverage on political repression. \newline
\end{minipage}
\vfill
\begin{minipage}{.49\linewidth}
  \centering
  \section*{\sf \underline{Attempted extension}}
  \raggedright
  \begin{itemize}
    \item {\bf Focus.} ER$_{t+1}$ performs relatively best.
    \item {\bf Improve consistency.} Use Civil Liberties dataset. 
    \item {\bf Jettison ballast.} $\boldsymbol{\tilde{X}} \subset \boldsymbol{X}: \boldsymbol{x} \perp \boldsymbol{\varepsilon}~\forall~\boldsymbol{x} \in  \boldsymbol{\tilde{X}}$ 
    \item {\bf Simplify.} $\textbf{ER}_{t+1}=\boldsymbol{\tilde{X}\beta}+\boldsymbol{\varepsilon}$
    \item {\bf Be sceptical.} Is co-optation linear?
    \item {\bf Be precise.} ER$_{t+1}\sim$\,Co-opt.|Physical integr. viol.
    \vspace{2.7em}
  \end{itemize}
\end{minipage}
\hfill
\begin{minipage}{.49\linewidth}
  \bigskip
  \centering
  \section*{\sf \underline{Cross-validation by country}}
  \includegraphics[scale=1.1]{/home/dag/Dropbox/Buero/Dissertation/2015/duke/classes/mle/termPaper/out/cvCountryCoef_manual.pdf} \\
  \raggedright
\end{minipage}

}
\headerbox{\sf \color{white} Simulation and out-of-sample prediction}{name=simulation,column=3,row=0,textborder=roundedleft}{
  \begin{minipage}{\linewidth}
  \centering
  \includegraphics[width=3in]{/home/dag/Dropbox/Buero/Dissertation/2015/duke/classes/mle/termPaper/out/spaghettiInteraction.pdf} \\
  \centering
  \raggedright
  Statistical simulations point to the assumed conditionality: The expected value of ER$_{t+1}$ decreases as 
  co-optation increases and physical integrity violations counterbalance this effect. Yet, already accounting for 
  systematic uncertainty proves the extended model unreliable.
  \end{minipage}
  \hfill
  \begin{minipage}{\linewidth}
  \centering
  \includegraphics[width=3in]{/home/dag/Dropbox/Buero/Dissertation/2015/duke/classes/mle/termPaper/out/testSamplePred.pdf} \\
  \raggedright
  Twenty percent of the $138$ countries were withheld for 
  out-of-sample validation. Given an in-sample RMSE of 
  $1.05$ and a test-sample score of $1.10$ the extended
  model seems to perform well. Alas, within-country variance 
  in ER$_{t+1}$ is overestimated -- sometimes dramatically.  
  \end{minipage}
}
\headerbox{\sf \color{white} Conclusion}{name=conclusions,column=3,,textborder=roundedleft, below=simulation}{
  Dictators maintain power by a combination of co-optation and
  political repression. Investigating their interaction is a 
  venue for future research. Nevertheless, statistical 
  models and data should be chosen and evaluated more carefully.
}
\headerbox{\sf \color{white} Reference}{name=references,column=3,below=conclusions,textborder=roundedleft}{
  Frantz, Erica \& Andrea Kendall-Taylor (2014) A dictator's toolkit:
    Understanding how co-optation affects repression in autocracies.
    {\it Journal of Peace Research}, 51(3): 332-346.
}
\headerbox{\sf \color{white} Partner up!}{name=references,column=3,below=references,textborder=roundedleft,bottomaligned=first}{
  \begin{minipage}{.29\linewidth}
    \centering
    {\sf GitHub} \\
    \includegraphics[scale=.18]{qrcodeGithub.png}
  \end{minipage}
  \hfill
  \begin{minipage}{.29\linewidth}
    \centering
    {\sf Contact} \\
    \includegraphics[scale=.18]{qrcodeWzb.png}
  \end{minipage}
  \hfill
  \begin{minipage}{.38\linewidth}
    \centering
    \includegraphics[scale=.4]{logo_en.png}
  \end{minipage}
}
\end{poster}
\end{document}