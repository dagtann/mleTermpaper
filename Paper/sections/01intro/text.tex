\begin{wrapfigure}{r}{0.45\textwidth}
\centering
\includegraphics[width=1\linewidth]{./sections/01intro/worldmapTermpaperIntro.pdf}
\caption{Parties and legislatures in authoritarian regimes, 2004}
\label{fig:worldmapIntro}
\end{wrapfigure}
Contemporary research holds that co-optation and political 
repression represent two mainstays of authoritarian regimes 
\citep[21f.]{Gerschewski.2013}. Usually co-optation 
is defined as ``the intentional extension of benefits to 
potential challengers to the regime in exchange for their 
loyalty'' \citep[333]{Frantz.2014}. Legislatures and 
political parties are said to simplify those exchanges. Since 
the end of the Cold War those nominally democratic 
institutions have taken root in almost every authoritarian 
regime. In fact, by the end of the observation period of the
replication study (2004) only Saudi Arabia, Oman, the 
United Arab Emirates, and Qatar sustained neither political 
parties nor a publicly elected parliament. At the same time 
authoritarian regimes did not forget about political 
repression. Restrictions on core political liberties 
and violations of physical integrity rights limit public 
criticism of the government and undermine coordinated 
campaigns against it. Yet, little is know about how 
co-optation affects political repression.

This is the point of departure for Erica Frantz' and Andrea 
Kendall-Taylor's \citeyearpar{Frantz.2014} `A dictator’s 
toolkit: Understanding how co-optation affects repression in 
autocracies'. Based on extensive quantitative analyses they 
argue that co-optation fundamentally changes the use of 
repression \citep[332]{Frantz.2014}. More precisely, they 
find that increasing levels of co-optation lead dictators 
to reduce restrictions on empowerment rights, but at the 
same time they increase physical integrity violations. The 
authors explain their key finding with the trade-offs 
involved in political repression. Restrictions on 
empowerment such as the freedoms of speech and assembly aim 
at the general public and characterize a diffuse approach to
social control. Physical integrity violations such as 
torture and extra-judicial killings in contrast target 
specific individuals and are more attractive when the 
opposition is known. Nominally democratic institutions offer
fora where regime opponents can raise demands and
thus they increase the available information on the 
political opposition. Under the bottom line, the 
institutions of co-optation generate knowledge on threats to
the regime and lead dictators to prefer physical integrity 
violations over empowerment rights restrictions 
\citep[337]{Frantz.2014}.

This paper replicates the work of Frantz and Kendall-Taylor.
It presents evidence on the violation of key statistical assumptions in the original publication and raises concerns 
with regard to predictive accuracy. Moreover, it casts doubt 
on a widespread estimation strategy that depends on lagged 
dependent variables to control for serial autocorrelation in
pooled time-series cross-sectional designs. My own extension
considers the possibility that increases in physical 
integrity violations undermine the credibility of nominally
democratic political institutions and attenuate the emancipating effect they might have on empowerment rights restrictions. The following section describes design and data 
and design of the original publication, and section three 
presents the replication results. Section four discusses my 
modified model, and section five concludes.