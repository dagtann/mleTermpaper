Based on Geddes et al.'s \citeyearpar{Geddes.2014}
``Autocratic regimes'' data Frantz and Kendall-Taylor
analyze 154 dictatorships over the period from 1981 to 2004.
The authors follow the example of \citet{Vreeland.2008} and
run ordered logistic regressions
(c.f. \cite{Fox.2008,Fox.2011}) to account for the ordinal 
nature of their dependent variables. Consequently, their 
research design probes the effect of co-optation on either 
type of political repression, empowerment rights 
restrictions and physical integrity violations, based on 
pooled time-series cross-section data. Furthermore, as 
institutional changes might take years to impact government 
policies, Frantz and Kendall-Taylor use contemporaneous 
levels of co-optation ($t_0$) to predict future levels 
of political repression ($t_0+1$ to $t_0+5$). All models 
include a lagged dependent variable ($t_0$) to account for 
serial autocorrelation and standard errors are clustered at 
the country level as a remedy to heteroscedasticity 
\citep{Beck.1995}. Finally, Frantz and Kendall-Taylor used
multiple imputation to avoid inefficiency and biased 
estimates or inference 
\citep{King.2001b,Honaker.2010,Honaker.2011}.

Information on political repression is drawn from two 
different sources. To assess the level of empowerment 
rights restrictions the authors rely on Freedom House's 
civil liberties scale. It captures the extent to which 
citizens enjoy the ``freedoms of expression and belief, 
associational and organizational rights, rule of law, and 
personal autonomy from the state'' 
\citep{FreedomHouse.2010}. In contrast to alternative 
measurements, Frantz and Kendall-Taylor argue, the Freedom 
House data is not endogenous to the existence of political 
parties and legislatures, i.e. their measurement of 
co-optation. The scale runs from $1$ to $7$, and higher 
values denote more restrictions on empowerment rights. 
Physical integrity violations are measured using the 
physical integrity index from the CIRI human rights dataset 
which provides ``standards-based measures of government 
human rights practices'' \citep[402]{Cingranelli.2010b}. 
It assesses the extent of torture, political imprisonment, 
extra-judicial killings, and disappearances on a scale from
$0$ to $8$ whereby higher values denote more government respect 
for the sanctity of person. Frantz and 
Kendall-Taylor recode the index such that higher values 
denote more political repression.

\begin{figure}[!htb]
\centering
\includegraphics[width=\linewidth]{./sections/02data/scatterRepression.pdf}
\caption{Political repression in authoritarian 
regimes between 1981 and 2004 with LOESS smoother and .95 
per cent confidence envelope added.}
\label{fig:scatterRepression}
\end{figure}

The typology of political repression draws out meaningful
differences between authoritarian regimes. This can be seen 
from Figure \ref{fig:scatterRepression} which explores 
their relationship in the unimputed data. The full range of
physical integrity violations is observed, but empowerment 
rights restrictions do not take their lowest possible value
$1$. Hence, all authoritarian regimes restrict civil 
and political liberties, but they do not always disrespect 
the sanctity of the individual at the same time. Moreover, 
Pearson's $r$ between both repression types is only  
$0.31$, and the LOESS smoother indicates that this already 
weak relationship disappears in certain regions of the data.
More precisely, the smoother stays flat across the most 
densely populated interval of empowerment rights 
restrictions ($4$ to $6$) and no inferences whatsoever may 
be drawn from changes in one type of political repression on 
the other. Consequently,  although authoritarian regimes
use both types of political repression there is empirical 
reason to believe that they differ to ``the extent to
which they rely on one type more than the other'' 
\citep[336]{Frantz.2014}.

Frantz and Kendall-Taylor assume that co-optation tips the 
scales of political repression. They measure this key 
explanatory variable by the existence of political parties 
and legislatures. Information on either institution is drawn 
from the `Democracy \& Dictatorship' data 
\citep{Cheibub.2010} that map their de facto existence. 
Frantz and Kendall-Taylor create an index that takes the 
value of 3 if there is a multi-party legislature, 
2 if there is a single-party legislature, 1 if there is no 
legislature but at least one political party or, 
equivalently, if there is a non-partisan legislature, and 0 
if neither exists. The authors presume that their index 
behaves linearly, and they justify their coding scheme with 
an interest in the ``interactive effect'' of legislatures 
and political parties \citep[338]{Frantz.2014}. Figure 
\ref{fig:barCooptation} explores the empirical picture in 
the unimputed data. The majority of $2,221$ non-missing 
country-year observations falls into the highest 
category. Accordingly, more than half of all authoritarian
regimes in the data sponsored multi-party legislatures. 
Single-party regimes come in second, and only a minority of
observations ranks lower than $2$ on the index. In sum, 
the crucial empirical distinction is whether authoritarian 
regimes co-opt via single party or multiple parties.

\begin{figure}[!htb]
\centering
\includegraphics[width=\linewidth]{./sections/02data/barCooptation.pdf}
\caption{Co-optation in authoritarian regimes 
  between 1981 2004, absolute frequencies.
}
\label{fig:barCooptation}
\end{figure}

To account for alternative explanations of political 
repression Frantz and Kendall-Taylor include a large set
of controls. Among these are counts of ongoing civil and 
interstate war as well as domestic political dissent in the 
form of riots, general strikes, and anti-government 
demonstrations. Moreover, the authors include counts of past
leadership turnovers and attempted coups under the assumption 
that authoritarian regimes with a history of leadership 
instability are more willing to repress. Another set of 
controls maps socio-economic conditions and historical 
context of the regime. For instance, assuming that 
oil-revenues offer alternative ways of co-optation Frantz 
and Kendall-Taylor control for oil rents per capita. 
Moreover, since size and growth of the population have been 
discussed as potential causes for state repression in the 
past the authors control for those as well. Moreover, they
add indicators on trade and economic well-being as well as
regime type. Moreover, to account for its considerable 
geopolitical repercussions a Cold War dummy is added to the 
model. Finally, following the advice of \citet{Carter.2010} 
cubic splines of leadership duration are added. Summary 
statistics of all controls variables are given in appendix A
(c.f. \cite[338f.]{Frantz.2014}).