Covering the period from 1972 to 2007 Frantz and 
Kendall-Taylor analyze 138 dictatorships based on the 
``Autocratic regimes'' data. The authors follow the example
set by \citet{Vreeland.2008} and run ordered logistic
regressions (c.f. \cite{Fox.2008,Fox.2011}) to account for
the ordinal characteristics of their dependent variables.
Their research design inquires into the effect of 
co-optation on either type of political repression, 
empowerment rights restrictions and physical integrity 
violations, based on pooled time-series cross-section data.
Furthermore, as institutional changes might take 
years before they impact government policies, Frantz and 
Kendall-Taylor use current levels of co-optation including
a set of control variables ($t_0$) to predict future levels 
of political repression ($t_0+1$ to $t_0+5$). All models 
include a lagged dependent variable ($t_0$) to account for 
serial autocorrelation and standard errors are clustered at 
the country level as a remedy to heteroscedasticity 
\citep{Beck.1995}. Finally, Frantz and Kendall-Taylor used
multiple imputation to fill gaps in the raw data and to 
avoid inefficiency as well as biased estimates or inference 
[KING ET AL CITATIONS].





