Authoritarian regimes maintain power via co-optation and 
political repression. Contemporary research has long 
recognized either as a pillar of authoritarian rule, but 
little is known about their mutual influence. This is the 
point of departure for Erica Frantz and Andrea 
Kendall-Taylor who argue that co-optation in the form of 
political parties and legislatures generates knowledge on 
the strength of the opposition such that dictators come to 
prefer targeted physical integrity violations over diffuse 
empowerment rights restrictions. My replication of their 
work uncovers the violation of a key statistical assumption,
it draws out the weak predictive accuracy of their analyses,
and it hints overfitting. 

My attempted extension of the original contribution 
emphasizes intuition over statistical complexity. I argue 
that the liberating effect of co-optation on empowerment 
rights is conditional on the level of physical integrity
violations. As dictators engage increasingly in torture 
institutionalized co-optation should lose its bite. However,
although the assumed interaction effect is statistically 
significant it lacks substantive significance and does not 
generalize from the training set. Moreover, my extension 
shows that the existence of political parties and 
legislatures does not seamlessly translate into a linear 
index of co-optation. Future revisions should, first, 
improve on the measurements involved and, second, employ 
statistically more sophisticated models such as longitudinal
multilevel regression.