%% --- Preamble -------------------------------------------------
\documentclass[parskip=half]{scrartcl}\usepackage[]{graphicx}\usepackage[]{color}
%% maxwidth is the original width if it is less than linewidth
%% otherwise use linewidth (to make sure the graphics do not exceed the margin)
\makeatletter
\def\maxwidth{ %
  \ifdim\Gin@nat@width>\linewidth
    \linewidth
  \else
    \Gin@nat@width
  \fi
}
\makeatother

\definecolor{fgcolor}{rgb}{0.345, 0.345, 0.345}
\newcommand{\hlnum}[1]{\textcolor[rgb]{0.686,0.059,0.569}{#1}}%
\newcommand{\hlstr}[1]{\textcolor[rgb]{0.192,0.494,0.8}{#1}}%
\newcommand{\hlcom}[1]{\textcolor[rgb]{0.678,0.584,0.686}{\textit{#1}}}%
\newcommand{\hlopt}[1]{\textcolor[rgb]{0,0,0}{#1}}%
\newcommand{\hlstd}[1]{\textcolor[rgb]{0.345,0.345,0.345}{#1}}%
\newcommand{\hlkwa}[1]{\textcolor[rgb]{0.161,0.373,0.58}{\textbf{#1}}}%
\newcommand{\hlkwb}[1]{\textcolor[rgb]{0.69,0.353,0.396}{#1}}%
\newcommand{\hlkwc}[1]{\textcolor[rgb]{0.333,0.667,0.333}{#1}}%
\newcommand{\hlkwd}[1]{\textcolor[rgb]{0.737,0.353,0.396}{\textbf{#1}}}%

\usepackage{framed}
\makeatletter
\newenvironment{kframe}{%
 \def\at@end@of@kframe{}%
 \ifinner\ifhmode%
  \def\at@end@of@kframe{\end{minipage}}%
  \begin{minipage}{\columnwidth}%
 \fi\fi%
 \def\FrameCommand##1{\hskip\@totalleftmargin \hskip-\fboxsep
 \colorbox{shadecolor}{##1}\hskip-\fboxsep
     % There is no \\@totalrightmargin, so:
     \hskip-\linewidth \hskip-\@totalleftmargin \hskip\columnwidth}%
 \MakeFramed {\advance\hsize-\width
   \@totalleftmargin\z@ \linewidth\hsize
   \@setminipage}}%
 {\par\unskip\endMakeFramed%
 \at@end@of@kframe}
\makeatother

\definecolor{shadecolor}{rgb}{.97, .97, .97}
\definecolor{messagecolor}{rgb}{0, 0, 0}
\definecolor{warningcolor}{rgb}{1, 0, 1}
\definecolor{errorcolor}{rgb}{1, 0, 0}
\newenvironment{knitrout}{}{} % an empty environment to be redefined in TeX

\usepackage{alltt}

%% --- PDF setup ------------------------------------------------
\usepackage[
  pdftex, 
  pdfpagelabels=false, 
  bookmarksopenlevel=section
]{hyperref}
\hypersetup{
  pdftitle = {MLE term paper}
  pdfauthor = {Dag Tanneberg},
  bookmarksnumbered = true,
  bookmarksopen = false,
  colorlinks = true,
  linkcolor = blue,
  citecolor = blue,
  urlcolor = blue
}

%% --- Page setup -----------------------------------------------
\usepackage{scrpage2}                                  %% Headers
\lohead{Dag Tanneberg}
\cohead{MLE term paper}
\rohead{\today}
\cfoot{\bfseries\pagemark}
\pagestyle{scrheadings}

\clubpenalty = 10000 %                                no orphants
\widowpenalty = 10000 \displaywidowpenalty = 10000    % no widows

\usepackage[hang]{footmisc}
\usepackage{lineno}
\usepackage{booktabs}
\usepackage{multirow}

\usepackage{setspace}
\usepackage{afterpage}

%% --- Symbols, graphics, and blindtext -------------------------
\usepackage{bbm}                     %% indicator function symbol
\usepackage{amsmath}
\usepackage{amssymb}
\usepackage{amsthm}
\newtheorem{prop}{Proposition}

\usepackage{graphicx}
\usepackage{wrapfig}

\usepackage{tikz}
  \usetikzlibrary{calc}
  \usetikzlibrary{matrix}
  \usetikzlibrary{positioning}

\usepackage{color}
\usepackage{xcolor}
  \definecolor{darkgrey}{HTML}{636363}
  \definecolor{lightgrey}{HTML}{F0F0F0}

\usepackage{blindtext}
% --- Library setup ---------------------------------------------
\usepackage[
    backend=biber,
    style=authoryear-icomp,
    sortlocale=en_EN,
    natbib=true,
    url=false, 
    doi=false,
    isbn=false,
    eprint=false
]{biblatex}
\addbibresource{./library.bib}

%% --- Author and document information --------------------------
\title{POLSCI.733\\Maximum likelihood estimation}
\subtitle{Term paper}
\author{Dag Tanneberg\thanks{%
    dag.tanneberg@duke.edu
  }
}
\date{\today}

%% --- Document body --------------------------------------------
\IfFileExists{upquote.sty}{\usepackage{upquote}}{}
\begin{document} 
\linenumbers

\maketitle
\thispagestyle{empty}
\tableofcontents
\newpage
%\onehalfspacing

\section{Introduction}






Authoritarian regimes maintain power via co-optation and 
political repression. Contemporary research has long 
recognized either as a pillar of authoritarian rule, but 
little is known about their mutual influence. This is the 
point of departure for Erica Frantz and Andrea 
Kendall-Taylor who argue that co-optation in the form of 
political parties and legislatures generates knowledge on 
the strength of the opposition such that dictators come to 
prefer targeted physical integrity violations over diffuse 
empowerment rights restrictions. My replication of their 
work uncovers the violation of a key statistical assumption,
it draws out the weak predictive accuracy of their analyses,
and it hints overfitting. 

My attempted extension of the original contribution 
emphasizes intuition over statistical complexity. I argue 
that the liberating effect of co-optation on empowerment 
rights is conditional on the level of physical integrity
violations. As dictators engage increasingly in torture 
institutionalized co-optation should lose its bite. However,
although the assumed interaction effect is statistically 
significant it lacks substantive significance and does not 
generalize from the training set. Moreover, my extension 
shows that the existence of political parties and 
legislatures does not seamlessly translate into a linear 
index of co-optation. Future revisions should, first, 
improve on the measurements involved and, second, employ 
statistically more sophisticated models such as longitudinal
multilevel regression.

\section{Design \& data}




Authoritarian regimes maintain power via co-optation and 
political repression. Contemporary research has long 
recognized either as a pillar of authoritarian rule, but 
little is known about their mutual influence. This is the 
point of departure for Erica Frantz and Andrea 
Kendall-Taylor who argue that co-optation in the form of 
political parties and legislatures generates knowledge on 
the strength of the opposition such that dictators come to 
prefer targeted physical integrity violations over diffuse 
empowerment rights restrictions. My replication of their 
work uncovers the violation of a key statistical assumption,
it draws out the weak predictive accuracy of their analyses,
and it hints overfitting. 

My attempted extension of the original contribution 
emphasizes intuition over statistical complexity. I argue 
that the liberating effect of co-optation on empowerment 
rights is conditional on the level of physical integrity
violations. As dictators engage increasingly in torture 
institutionalized co-optation should lose its bite. However,
although the assumed interaction effect is statistically 
significant it lacks substantive significance and does not 
generalize from the training set. Moreover, my extension 
shows that the existence of political parties and 
legislatures does not seamlessly translate into a linear 
index of co-optation. Future revisions should, first, 
improve on the measurements involved and, second, employ 
statistically more sophisticated models such as longitudinal
multilevel regression.

\section{Replication results}
% Uncomment to reproduce coefficient plot for
% <<coefMultinom, child = './sections/03replication/coefficientMultinomial.Rnw'>>=
% @
% NOTE: Labels in left-hand panel are not correct and must
% be changed externally. Lowest ER response level is >=4!
%
%<<separationPlot, child ='./sections/03replication/separation_modified.Rnw'>>=
%@
%
%<<bicDifferences, child='./sections/03replication/bicDifference.Rnw'>>=
%@
Authoritarian regimes maintain power via co-optation and 
political repression. Contemporary research has long 
recognized either as a pillar of authoritarian rule, but 
little is known about their mutual influence. This is the 
point of departure for Erica Frantz and Andrea 
Kendall-Taylor who argue that co-optation in the form of 
political parties and legislatures generates knowledge on 
the strength of the opposition such that dictators come to 
prefer targeted physical integrity violations over diffuse 
empowerment rights restrictions. My replication of their 
work uncovers the violation of a key statistical assumption,
it draws out the weak predictive accuracy of their analyses,
and it hints overfitting. 

My attempted extension of the original contribution 
emphasizes intuition over statistical complexity. I argue 
that the liberating effect of co-optation on empowerment 
rights is conditional on the level of physical integrity
violations. As dictators engage increasingly in torture 
institutionalized co-optation should lose its bite. However,
although the assumed interaction effect is statistically 
significant it lacks substantive significance and does not 
generalize from the training set. Moreover, my extension 
shows that the existence of political parties and 
legislatures does not seamlessly translate into a linear 
index of co-optation. Future revisions should, first, 
improve on the measurements involved and, second, employ 
statistically more sophisticated models such as longitudinal
multilevel regression.

\newpage
\appendix
\section{Summary statistics of controls}



To account for alternative explanations of political 
repression Frantz and Kendall-Taylor include a large set
of controls (c.f. \cite[338f.]{Frantz.2014}). Among these 
are counts of ongoing civil and interstate war as well as 
domestic political dissent in the form of riots, general 
strikes, and anti-government demonstrations. Moreover, the 
authors include counts of past leadership turnovers and 
attempted coups under the assumption that authoritarian 
regimes with a history of leadership instability are more 
willing to repress. Other controls map the socio-economic 
status and historical context of the regime. For instance, 
assuming that oil-revenues offer alternative ways of 
co-optation Frantz and Kendall-Taylor control for oil rents 
per capita. Moreover, since size and growth of the 
population have been discussed as potential causes for state
repression in the past the authors control for those as 
well. Moreover, they add indicators on trade and economic 
well-being as well as regime type. Moreover, to account for 
its considerable geopolitical repercussions a Cold War dummy
is added to the model. Finally, following the advice of 
\citet{Carter.2010} cubic splines of leadership duration are
added. 


% Table created by stargazer v.5.1 by Marek Hlavac, Harvard University. E-mail: hlavac at fas.harvard.edu
% Date and time: Mi, Apr 29, 2015 - 19:46:48
\begin{table}[!htbp] \centering 
  \caption{Summary statistics of control variables} 
  \label{} 
\begin{tabular}{@{\extracolsep{5pt}}lccccc} 
\\[-1.8ex]\hline \\[-1.8ex] 
Statistic & \multicolumn{1}{c}{Min} & \multicolumn{1}{c}{Mean} & \multicolumn{1}{c}{Max} & \multicolumn{1}{c}{St. Dev.} & \multicolumn{1}{c}{N} \\ 
\hline \\[-1.8ex] 
Civil war & 0 & 0.240 & 5 & 0.601 & 2,386 \\ 
Interstate war & 0 & 0.063 & 2 & 0.250 & 2,386 \\ 
log(population) & 4.215 & 8.777 & 14.074 & 1.712 & 2,352 \\ 
log(GDP per capita) & 5.139 & 7.913 & 10.807 & 1.058 & 2,352 \\ 
Personal regime & 0 & 0.292 & 1 & 0.455 & 1,857 \\ 
Monarchy & 0 & 0.097 & 1 & 0.297 & 1,857 \\ 
Dominant party regime & 0 & 0.489 & 1 & 0.500 & 1,857 \\ 
Trade (% of GDPpc) & 0.309 & 76.026 & 423.568 & 45.332 & 2,024 \\ 
Cold War & $-$50.046 & 1.003 & 90.470 & 7.694 & 2,049 \\ 
Growth (% of GDPpc) & 0 & 10.827 & 47 & 9.513 & 2,271 \\ 
Leadership duration & 0 & 4.379 & 43 & 6.471 & 2,386 \\ 
Past leadership fails & 0 & 2.264 & 22 & 3.004 & 2,386 \\ 
Past coups & $-$11.513 & $-$3.867 & 10.811 & 8.328 & 2,250 \\ 
Oil rents & 0 & 0.090 & 5 & 0.442 & 1,857 \\ 
Election year & 0 & 0.358 & 23 & 1.378 & 1,857 \\ 
Strikes & 0 & 0.634 & 26 & 2.034 & 1,857 \\ 
\hline \\[-1.8ex] 
\end{tabular} 
\end{table} 


\newpage
\addcontentsline{toc}{section}{References}
\printbibliography
\end{document}
